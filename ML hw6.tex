\documentclass[10pt]{article}
\oddsidemargin -.2in
\evensidemargin -.2in
\topmargin -.2in
\textwidth 6.9in
\textheight 8.7in
%%%%%%%%%%%%%%%%%%%%%%%%%

\usepackage[parfill]{parskip}    % Activate to begin paragraphs with an empty line rather than an indent
\usepackage{latexsym, amssymb, bbm}
\usepackage{amsmath}
\usepackage{amsfonts}
\usepackage{amsthm}
\usepackage{verbatim}
\usepackage{fancyhdr}
\usepackage{enumerate}
\usepackage{mathtools}
\usepackage{graphicx}
\usepackage{float}
\usepackage[linesnumbered,vlined]{algorithm2e}
\SetEndCharOfAlgoLine{}
\usepackage{changepage}

%%%%%%%%%%%%%%%%%%%%%%%%%

\newtheorem{thm}{Theorem}[section]
\newtheorem{lem}[thm]{Lemma}
\newtheorem{prop}[thm]{Proposition}
\newtheorem{cor}[thm]{Corollary}
\newtheorem{cla}[thm]{Claim}
\newtheorem{con}[thm]{Conjecture}

\theoremstyle{definition}
\newtheorem*{defn}{Definition}
\newtheorem{example}[thm]{Example}
\newtheorem{axiom}{Axiom}
\newtheorem*{note}{Note}
\newtheorem*{question}{Question}

\theoremstyle{remark}
\newtheorem{case}{Case}

%%%%%%%%%%%%%%%%%%%%%%%%%
%%%%%%%%%%%%%%%%%%%%%%%%%%
%%%%%%%% EDIT HERE %$%%%%%%%%%
%%%%%%%%%%%%%%%%%%%%%%%%%%%%
%%%%%%%%%%%%%%%%%%%%%%%%%%%%%

\newcommand{\name}{Casper Neo}
\newcommand{\classname}{Machine Learning}
\newcommand{\classnumber}{CMSC 254}
\newcommand{\due}{28 February 2017}
\newcommand{\num}{5}
\newcommand{\type}{Homework}

%%%%%%%%%%%%%%%%%%%%%%%%%%%%%
%%%%%%%%%%%%%%%%%%%%%%%%%%%%
%%%%%%%%%%%%%%%%%%%%%%%%%%%
%%%%%%%%%%%%%%%%%%%%%%%%%%
%%%%%%%%%%%%%%%%%%%%%%%%%


%-------------------------------------------------------


%-------------------------------------------------------

\newcommand{\lecture}[1]{\centerline{\fbox{\textbf{#1}}}}
\newcommand{\que}[2] {\vspace{.25in} \fbox{#1} #2 \vspace{.10in}}
\renewcommand{\part}[1] {\vspace{.10in} {\bf (#1)}}


%%%%%%%%%%%%%%%%%%%%%%%%%%

% SHORTCUTS FOR BLACKBOARD BOLD
\def\C{\mathbb{C}}
\def\N{\mathbb{N}}
\def\Q{\mathbb{Q}}
\def\R{\mathbb{R}}
\def\Z{\mathbb{Z}}
\def\1{\mathbbm{1}}
\def\P{\mathbb{P}}
\def\0{\emptyset}

% SHORTCUTS FOR MATH BOLD FONT
\def\ba{\mathbf{a}}
\def\bb{\mathbf{b}}
\def\be{\mathbf{e}}
\def\bm{\mathbf{m}}
\def\bh{\mathbf{h}}
\def\bu{\mathbf{u}}
\def\bw{\mathbf{w}}
\def\bv{\mathbf{v}}
\def\bS{\mathbf{\Si}}
\def\bmu{\boldsymbol{\mu}}
\def\bx{\mathbf{x}}
\def\by{\mathbf{y}}
\def\bz{\mathbf{z}}
\def\bzero{\mathbf{0}}
\def\bone{\mathbf{1}}

% SHORTCUTS FOR GREEK LETTERS
\def\a{\alpha}
\def\b{\beta}
\def\g{\gamma}
\def\G{\Gamma}
\def\d{\delta}
\def\D{\Delta}
\def\e{\varepsilon}
\def\k{\kappa}
\def\l{\lambda}
\def\m{\mu}
\def\r{\rho}
\def\s{\sigma}
\def\t{\tau}
\def\th{\theta}
\def\x{\xi}
\def\vp{\varphi}
\def\n{\nabla}
\def\w{\omega}
\def\z{\zeta}
\def\vp{\varphi}
\def\p{\phi}
\def\Ph{\Phi}
\def\L{\Lambda}
\def\Si{\Sigma}
\def\Th{\Theta}
\def\O{\Omega}

\def\el{\ell}

% SHORTCUTS FOR MATH SCRIPT
\def\sR{\mathscr{R}}
\def\sA{\mathscr{A}}
\def\sC{\mathscr{C}}
\def\sM{\mathscr{M}}
\def\sF{\mathscr{F}}
\def\sB{\mathscr{B}}
\def\sL{\mathscr{L}}

% SHORTCUTS FOR MATH CAPITALS
\def\A{\mathcal{A}}
\def\cC{\mathcal{C}}
\def\cE{\mathcal{E}}
\def\cF{\mathcal{F}}
\def\cH{\mathcal{H}}
\def\cL{\mathcal{L}}
\def\cM{\mathcal{M}}
\def\cP{\mathcal{P}}
\def\cU{\mathcal{U}}


\def\i{^{-1}}
\newcommand{\fin}{f^{-1}}
\def\lp{\mathop{\rm lp}}
\def\slp{\mathop{\rm slp}}
\def\ext{\mathop{\rm ext}}
\def\sep{\mathop{\rm sep}}
\def\rel{\mathop{\rm rel}}
\def\lub{\mathop{\rm lub}}
\def\glb{\mathop{\rm glb}}

\def\ds{\displaystyle}

\newcommand{\lpvar}[1]{\,\textrm{lp}_{#1}\,}
\newcommand{\ol}[1]{\overline{#1}}
\newcommand{\fl}[1]{\left\lfloor #1 \right\rfloor}
\newcommand{\cl}[2]{\left\lceil \frac{#1}{#2} \right\rceil}
\newcommand{\md}[1]{(\text{mod } #1)}
\newcommand{\cs}[1]{\begin{case} #1 \end{case}}
\newcommand{\abs}[1]{\left\lvert #1 \right\rvert}
\newcommand{\norm}[1]{\lVert#1\rVert}
\newcommand{\set}[1]{\{ #1 \}}
\DeclarePairedDelimiter{\ceil}{\lceil}{\rceil}
\DeclarePairedDelimiter{\floor}{\lfloor}{\rfloor}


\newcommand{\upint}[2]{
  \overline{\int_{#1}^{#2}}
}
\newcommand{\loint}[2]{
  \underline{\int_{#1}^{#2}}
}

\DeclareMathOperator{\spn}{span}
\DeclareMathOperator{\diam}{diam}
\DeclareMathOperator*{\E}{\mathbb{E}}
\DeclareMathOperator*{\pr}{\mathbb{P}}

\newenvironment{piecewise}{\left\lbrace
\begin{matrix}}
{
\end{matrix}
\right.
}



\def\defeq{\stackrel{\text{\tiny def}}=}

\def\la{\langle}
\def\ra{\rangle}

\def\ben{\begin{enumerate}}
\def\een{\end{enumerate}}

\def\ptl{\partial}
\newcommand{\pd}[2]{\frac{\ptl#1}{\ptl #2}}

\def\tb{\textbf}
\def\^{\wedge}
\def\del{\nabla}

% Little Stats things
\def\iid{\stackrel{iid}\sim}
\def\indep{\!\perp\!\!\!\perp}


% SHORTCUTS FOR LIMITS
\def\ltx{\lim_{t\to x}}
\def\lxp{\lim_{x\to p}}
\def\lxa{\lim_{x\to a}}
\def\lni{\lim_{n\to\infty}}
\def\lsni{\limsup_{n\to\infty}}
\def\lxi{\lim_{x\to\infty}}
\def\lxz{\lim_{x\to0}}
\def\lhz{\lim_{h\to0}}
\def\ltz{\lim_{t\to0}}

% SHORTCUTS FOR SUMMATIONS
\def\sion{\sum_{i=1}^n}
\def\sjon{\sum_{j=1}^n}
\def\sjom{\sum_{j=1}^m}
\def\snoi{\sum_{n=1}^\infty}
\def\sioi{\sum_{i=1}^\infty}
\def\snzi{\sum_{n=0}^\infty}
\def\snzN{\sum_{n=0}^N}
\def\skoi{\sum_{k=1}^\infty}
\def\siok{\sum_{i=1}^k}
\def\skon{\sum_{k=1}^n}
\def\skzn{\sum_{k=0}^n}
\def\pion{\prod_{i=1}^n}
\def\sjok{\sum_{j=1}^k}
\def\siod{\sum_{i=1}^d}
\def\sjod{\sum_{j=1}^d}

% SHORTCUTS FOR INTEGRALS
\def\ipp{\int_{-\pi}^\pi}
\def\iii{\int_{-\infty}^{\infty}}
\def\izp{\int_0^\pi}
\def\iztp{\int_0^{2\pi}}
\def\izr{\int_0^r}
\def\izo{\int_0^1}
\def\ioi{\int_1^\infty}
\def\iab{\int_a^b}
\def\izi{\int_0^\infty}
\def\riea{\sR(\a)}
\newcommand{\eval}[2]{\Big\vert_{#1}^{#2}}

% SHORTCUTS FOR ENVIRONMENTS
\newcommand{\al}[1]{\begin{align*}#1\end{align*}}
\newcommand{\aleq}[1]{\begin{equation}\begin{aligned}[b]#1\end{aligned}\end{equation}}
\newcommand{\bigb}[1]{\bigg(#1\bigg)}
\newcommand{\prf}[1]{\begin{proof}#1\end{proof}\vspace{3mm}}
\newcommand{\bmtx}[1]{\begin{bmatrix}#1\end{bmatrix}}
\newcommand{\enum}[2]{\begin{enumerate}[#1]#2\end{enumerate}}
\newcommand{\pwise}[1]{\begin{piecewise}#1\end{piecewise}}
\newcommand{\inm}[1]{\norm{#1}_2}
\newcommand{\enviro}[2]{\begin{#1}#2\end{#1}}
\newcommand{\tabler}[2]{\begin{center}\begin{tabular}{#1}#2\end{tabular}\end{center}}

\newcommand{\seq}[2]{\{#1_#2\}}
\DeclareMathOperator*{\argmin}{arg\,min}
\DeclareMathOperator*{\argmax}{arg\,max}

%%%%%%%%%%%%%%%%%%%%%%%%%%%%%%%%%%%%%%%%%%%%%%%%

\begin{document}
%--------------------------------------------------------------
\pagestyle{fancy}

\lhead{\name}
\chead{\classnumber\space\space-\space\type\space\num}
\rhead{\due}

\title{
	\vspace{-20mm}
    \classnumber\\
	\classname\\
    \type \space \num\\
}
\author{
	\name \\\small
}
\author

\date{
	\due\\
}

\maketitle
%--------------------------------------------------------------

\begin{enumerate}
    \item \begin{enumerate}
        \item

        Consider the algorithm where for every letter in $\ol a$
        and for every letter in $\ol b$ we check if there is a $\k$ length
        string match. Then this clearly finds $k_{\k}(\ol a,\ol b)$ and
        the algorithm is $O( l_{\ol a} l_{\ol b} \k)$ upon examining the loops.
        \begin{center}
        \begin{minipage}{.9\textwidth}
        \begin{algorithm}[H]
            \KwData{Strings $\ol a,\ol b$, integer $\k$}
            \KwResult{ $k_{\k}(\ol a,\ol b)$}
            $n=0$\\
            \For{$i=0,i<l_{\ol a} - k, i++$}{
                \For{$j=0,j<l_{\ol b}-k, j++$}{
                    $substr = True$ \\
                    \For{$p=0,p<k,p++$}{
                        \If{$\ol a_{i+p} \not= \ol b_{j+p}$}{
                            $substr = False$ \\
                            \textbf{Break}
                        }
                    }
                    \lIf{substr}{n++}
                }
            }
            \Return{n}

        \end{algorithm}
        \end{minipage}
        \end{center}

        Observe that
        $k_\k(\ol a,\ol b) = \sum_{\ol s\in \Sigma^\k}n_{\ol s}(\ol a)n_{\ol s}(\ol b)$
        is the inner product over the hilbert space reached by the function
        $\phi(\ol a) \to (n_{s_1}(\ol a) ... n_{s_{\abs{\Sigma^\k}}}(\ol a))^T$
        hence $k_\k$ is positive semi definite.


        \item
        This kernel counts the number string matches between $\ol a$ and $\ol b$
        allowing for gaps in the strings. Analogously to part (a), this kernel
        corresponds to the Hilbert space reached by the function
        $\phi(\ol a) \to (n_{s_1}^*(\ol a) ... n^*_{s_{\abs{\Sigma^\k}}}(\ol a))^T$
        and so it inherits positive semi-definiteness from the inner product.

        \begin{enumerate}
            \item Consider the algorithm
            which has $O(\abs{\Sigma}(l_{\ol a} + l_{\ol b}))$ character
            comparisons.
            \begin{center}
            \begin{minipage}{.9\textwidth}
            \begin{algorithm}[H]
                \KwData{Strings $\ol a,\ol b$}
                \KwResult{ $k_1^*(\ol a,\ol b)$}
                $k = 0$\\
                \For{$s\in \Sigma$}{
                    $as = 1$; $bs = 1$ \\
                    \For{$i=0,i<l_{\ol a}, i++$}{
                        \lIf{$\ol a_i == s$}{
                            $as ++$
                        }
                    }
                    \For{$j=0,j<l_{\ol b},j++$}{
                        \lIf{$\ol b_j == s$}{
                            $bs ++$
                        }
                    }
                    $k = k + as \cdot bs$ \\
                }
                \Return k

            \end{algorithm}
            \end{minipage}
            \end{center}

            \item Observe $k^*_p(\ol a_{0:p-1},\ol b_{0:p-1})
                = \1(\ol a_{0:p-1} = \ol b_{0:p-1})$
                where $\1$ is the indicator function.
                Hence the following 6 lines of code return the result in $O(p)$
                checks.
                \begin{center}
                \begin{minipage}{.9\textwidth}
                \begin{algorithm}[H]
                    \KwData{Strings $\ol a,\ol b$}
                    \KwResult{ $k_p^*(\ol a_{0:p-1},\ol b_{0:p-1})$}
                    $k = 1$\\
                    \For{$i=0,i<p,i++$}{
                        \If{$\ol a_i \not= \ol b_i$}{
                            $k=0$\\
                            \textbf{Break}
                        }
                    }

                    \Return k

                \end{algorithm}
                \end{minipage}
                \end{center}


            \item \begin{enumerate}
                \item Observe $k_p^*$ is the number of gappy substring of length $p$
                pairs between $\ol a$ and $\ol b$. So finding
                $k_p^*(\ol a_{0:N+1},\ol b_{0:j})$ is just
                $k_p^*(\ol a_{0:N},b_{0:j})$ plus the number of new gappy
                substring of length $p$ pairs. Since we're adding $\ol a_{n+1}$,
                when considering the substrings of length $p-1$ in $\ol b_{0,i}$
                we need to add any matches between $\ol b_{l}$ and $\ol a_{N+1}$
                where $l$ is between $i+1$ and $j$. Observe
                $k_{p-1}^*(\ol a_{0:N},\ol b_{0,i}) -
                 k_{p-1}^*(\ol a_{0:N},\ol b_{0,i-1})$ is the number of gappy
                 substring of length $p-1$ matches that end at $b_i$.
                 Therefore we have the formula
                $$
                    k_p^*(\ol a_{0:N+1},\ol b_{0:j}) =
                    k_p^*(\ol a_{0:N},b_{0:j}) +
                    \sum_{i=0}^{j-1}
                    (k_{p-1}^*(\ol a_{0:N},\ol b_{0:i})
                    -k_{p-1}^*(\ol a_{0:N},\ol b_{0:i-1}))
                    \sum_{l = i+1}^{j}
                    \1(
                        \ol b_{l} = \ol a_{N+1}
                    )
                $$

                \item Analogously to part $A.$ but exchanging $\ol a$ and $\ol b$.


                $$
                    k_p^*(\ol a_{0:i},\ol b_{0:M+1}) =
                    k_p^*(\ol a_{0:i},b_{0:M}) +
                    \sum_{j=0}^{i-1}
                    (k_{p-1}^*(\ol a_{0:j},\ol b_{0:M})
                    -k_{p-1}^*(\ol a_{0:i-1},\ol b_{0:M}))
                    \sum_{l = j+1}^{i}
                    \1(
                        \ol a_{l} = \ol b_{M+1}
                    )
                $$


                \item Combining the previous parts we get an algorithm which
                examining the loops is $O(l_{\ol a}l_{\ol b}k)$ in character
                checks.
                \begin{center}
                \begin{minipage}{.9\textwidth}
                \begin{algorithm}[H]
                    \KwData{Strings $\ol a,\ol b$, integer $\k$}
                    \KwResult{ $k_\k^*(\ol a,\ol b)$}
                     Let $A$ be a $l_{\ol a}\times l_{\ol b}\times \k$ matrix \\
                     $n=0$\\
                     \For{$i=0,...,l_{\ol a}-1$}{
                        \For{$i=0,...,l_{\ol b}-1$}{
                            \lIf{$\ol a[i] = \ol b[j]$}{$n+=1$}
                            $A[i,j,1] = n$
                        }
                     }
                     \For{$k = 2,...,\k$}{
                        $A[0,0,k] = A[1,0,k] = A[0,1,k] = 0$\\
                        \For{$N=1,...,l_{\ol a}-1$}{
                            \For{$M = 1,..., l_{\ol b}-1$}{
                                $
                                    A[N+1,M,k] = A[N,M,k] + \\
                                    \sum_{n=0}^{M-1}
                                    (A[N,n,k-1] - A[N,n-1,k-1])
                                    \sum_{l=n+1}^M\1(\ol b[l] = \ol a[N+1])
                                $ \\
                                $
                                    A[N,M+1,k] = A[N,M,k] + \\
                                    \sum_{n=0}^{N-1}
                                    (A[n,M,k-1] - A[n-1,M,k-1])
                                    \sum_{l=n+1}^N\1(\ol b[M+1] = \ol a[l])
                                $ \\

                            }
                        }
                     }
                     \Return{$A[l_{\ol a},l_{\ol b},\k]$}

                \end{algorithm}
                \end{minipage}
                \end{center}


            \end{enumerate}
        \end{enumerate}



    \end{enumerate}
    \item
    The following follow from the definition of probability in graphical
    models and the relationships described by the graphs in question.
    $$
        \P(\bx) = \prod_{v\in V}\P(x_v | x_p : p \in Parents(v))
    $$
    \begin{enumerate}
        \item Since $\P(X_a,X_b,X_c) = \P(X_a|X_c)\P(X_b|X_c)\P(X_c)$:
        \al{
            \P(X_a,X_b) &= \sum_{x_c}\P(X_a,X_b,X_c=x_c) \\
                & = \sum_{x_c} \P(X_a|x_c)\P(X_b|x_c)\P(x_c) \\
                &\not= \P(X_a)\P(X_b)
        }
        Therefore $ X_a \not\indep X_b$
        \al{
            \P(X_a,X_b | X_c)
                &= \frac{\P(X_a,X_b,X_c)}{\P(X_c)} \\
                &= \P(X_a|X_c)\P(X_b|X_c)
        }
        Therefore $(X_a \indep X_b) | X_c$.

        \item Since $\P(X_a,X_b,X_c) = \P(X_a)\P(X_b)\P(X_c | X_a,X_B)$
        \al{
            \P(X_a,X_b)
                &= \sum_{x_c}\P(X_a,X_b,X_c=x_c) \\
                &= \sum_{x_c}\P(X_a)\P(X_b)\P(X_c=x_c |X_a,X_b) \\
                &= \P(X_a,X_b)
        }
        Therefore $X_a\indep X_b$.
        \al{
            \P(X_a,X_b|X_c)
                &= \frac{\P(X_a,X_b,X_c)}{\P(X_c)}\\
                &= \frac{\P(X_a)\P(X_b)\P(X_c|X_a,X_b)}{\P(X_c)}  \\
                &\not= \P(X_a| X_c) \P(X_b | X_c)
        }
        Therefore $(X_a\not\indep X_b)|X_c$.

    \end{enumerate}

    \item $a,b$ are specified by 1 number each. $c,e$ depend on one other
    variable and thus are specified with 2 numbers each. $d$ depends on
    3 other variables and so requires 8 numbers. Hence the following graph may
    be specified with 14 parameters.

    \begin{figure}[H]
        \vspace{4cm}
        \center
        \includegraphics[width=.4\textwidth]{im1.jpg}

        \caption{My amazing drawing}
    \end{figure}

    \item \begin{itemize}
        \item $E\indep G | F$
            True since $G$ only depends on F, conditional on
            $F$ it is independent of everything.
        \item $A\indep C | B,G$
            False since information about $A$ and $C$ is captured in $G$ so $A$
            is partially specified by $C$
        \item $B\indep C$
            True since neither of them have parents.
        \item $A\indep B | C,G$
            False since information about $A$ and $B$ is captured in $G$ so $A$
            is partially specified by $B$


    \end{itemize}
    \item $$
        \P(a,b,c,d,e) = \P(a)\P(e)\P(b|a)\P(d|a,e)\P(c|b,d)
    $$

    This means \begin{itemize}
        \item $a\indep e$
        \item $e\indep b$
        \item $a\indep c|b,d$
        \item $e\indep c|d$
        \item $b\indep d|a$
    \end{itemize}

\end{enumerate}

%------------------------------------------------------------------------------------------------------------
\end{document}
